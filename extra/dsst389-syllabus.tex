\documentclass[11pt, a4paper]{article}

\usepackage{fontspec}
\usepackage{geometry}
\usepackage{fancyhdr}
\usepackage[hidelinks]{hyperref}
\usepackage[normalem]{ulem}
\usepackage{multicol}

\geometry{
  top=2cm,
  bottom=3cm,
  left=3cm,
  right=3cm,
  marginparsep=4pt,
  marginparwidth=1cm
}

\renewcommand{\headrulewidth}{0pt}
\pagestyle{fancyplain}
\fancyhf{}
\lfoot{}
\rfoot{}

\setlength{\parindent}{0pt}
\setlength{\parskip}{0pt}

\usepackage{xunicode}
\defaultfontfeatures{Mapping=tex-text}

\begin{document}

\begin{center}
\textbf{DSST389: Statistical Learning --- Taylor Arnold --- Spring 2023}
\end{center}

\vspace{0.5cm}

\textbf{Website}: \texttt{https://statsmaths.github.io/dsst389-s23}

\bigskip

\textbf{Topics:}
Methods for predictive modelling and dimensionality reduction, with
applications to computational text analysis.

\bigskip

\textbf{Readings \& Class Form:}
Each class meeting has a short reading posted on the course website.
Students are expected to do this reading before the next class.
Regular attendance is also expected. Excessive absences will be dealt
with by a warning followed by a reduction on the final course grade.
Attendance and reading completion is recorded by a form on the course
website submitted at the start of each class.

\bigskip

\textbf{Projects:}
There are four projects due during the semester, which may be completed in
small groups of two or three students. Grades given out of 95 points.

\bigskip

\textbf{Engagement:}
A reflection recording one's effort and engagement during the semester
is due on the last day of class. A grade will be given out of 95 points.

\bigskip

\textbf{Final Grades:}
The project and course engagement grades are averaged and a letter grade
is assigned as follows:
             A (93--95), A- (90--92),
B+ (87--89), B (83--86), B- (80--82),
C+ (77--79), C (73--76), C- (70--72), and F (0--69).


\end{document}