\documentclass[11pt, a4paper]{article}

\usepackage{fontspec}
\usepackage{geometry}
\usepackage{fancyhdr}
\usepackage[hidelinks]{hyperref}
\usepackage[normalem]{ulem}
\usepackage{multicol}

\geometry{
  top=2cm,
  bottom=3cm,
  left=3cm,
  right=3cm,
  marginparsep=4pt,
  marginparwidth=1cm
}

\renewcommand{\headrulewidth}{0pt}
\pagestyle{fancyplain}
\fancyhf{}
\lfoot{}
\rfoot{}

\setlength{\parindent}{0pt}
\setlength{\parskip}{0pt}

\usepackage{xunicode}
\defaultfontfeatures{Mapping=tex-text}

\begin{document}

\begin{center}
\textbf{DSST389: Statistical Learning --- Taylor Arnold --- Spring 2023}
\end{center}

\vspace{0.5cm}

\textbf{Website}: \texttt{https://statsmaths.github.io/dsst389-s23}

\bigskip

\textbf{Overview:}
Methods for collecting and modeling complex datasets, with applications
to network, temporal, text, and image analyses.

\bigskip

\textbf{Readings and Class Form:}
Most class meetings have a short reading posted on the course website.
Students are expected to do this reading before the next class. At the start
of each class, please fill out the course form posted on the website to indicate
whether you completed the reading and whether you are present in class. 
Regular attendance is expected. Excessive absences will be dealt with by a
warning followed by a reduction on the final course grade.

\bigskip

\textbf{Notebooks:}
Almost all class meetings have an associated R notebook.
These notebooks include a mixture of coding and short answer
responses. We will work on the notebooks together in class. Students are
expected to finish these on their own and to upload their solutions to a
course Box folder before the next course meeting. This folder will be shared
during the second week of the semester. Solutions to the coding questions
will be posted (and may be consulted and even copied) after class; open-ended
questions will (usually) not have any posted solutions. You may make
additional changes to your solutions up to two weeks after its due date.

\bigskip

\textbf{Course Reflection:}
A reflection recording one's effort and engagement during the semester
is due on the last day of class. This will include a question where you assign
yourself a final grade for the course, along with a justification.
A template of this form will be posted on the website.

\bigskip

\textbf{Final Grades:}
This course uses a holistic, portfolio-based grading method. A grade will
be assigned based on your class participation, course reflection, and your
uploaded notebooks. Once during the semester, around the fifth week of the
term, feedback will be provided to help guide set expectations and indicate
any areas that need improvement.

\end{document}