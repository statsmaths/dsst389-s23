\documentclass[10pt, a4paper]{article}

\usepackage{fontspec}
\usepackage{geometry}
\usepackage{lastpage}
\usepackage{fancyhdr}
\usepackage{amsmath, amsthm, amssymb}
\usepackage{url}
\usepackage[hidelinks]{hyperref}
\usepackage{changepage}
\usepackage{quoting}
\usepackage{xcolor}
\usepackage{array}

\quotingsetup{vskip=0pt}

\definecolor{solarized@magenta}{HTML}{D33682}
\newcommand{\magenta}[1]{\textcolor{solarized@magenta}{#1}}

\usepackage[absolute]{textpos}
\setlength{\TPHorizModule}{1in}
\setlength{\TPVertModule}{1in}

\newcommand{\minimize}{\mathop{\mathrm{minimize}}}
\newcommand{\argmin}{\mathop{\mathrm{arg\,min}}}
\newcommand{\argmax}{\mathop{\mathrm{arg\,max}}}
\newcommand{\st}{\mathop{\mathrm{subject\,\,to}}}

\geometry{
  top=1.5cm, bottom=2cm, left=1.5cm, right=1.5cm, marginparsep=4pt, marginparwidth=1in
}

\renewcommand{\headrulewidth}{0pt}
\pagestyle{fancyplain}
\fancyhf{}
\cfoot{}

\setlength{\parindent}{0pt}
\setlength{\parskip}{12pt}

\usepackage{marginnote} % For margin years
\newcommand{\years}[1]{\marginnote{\scriptsize #1}}
\renewcommand*{\raggedleftmarginnote}{}
\setlength{\marginparsep}{-16pt}
\reversemarginpar

\renewcommand\refname{}

\setromanfont[
 Ligatures={Common},
 Numbers={OldStyle},
 Variant=01,
 BoldFont={LinLibertine_RB.otf},
 ItalicFont={LinLibertine_RI.otf},
 BoldItalicFont={LinLibertine_RBI.otf}
]{LinLibertine_R.otf}

\usepackage{xunicode}
\defaultfontfeatures{Mapping=tex-text}
\usepackage{changepage}


\newcolumntype{C}{>{\raggedright\arraybackslash}p{0.13\linewidth}}
\newcolumntype{E}{>{\centering\arraybackslash}p{0.06\linewidth}}
\newcolumntype{L}{>{\centering\arraybackslash}m{0.13\linewidth}}

\begin{document}

\fontsize{8pt}{10pt}\selectfont

\textbf{Project Number}: 1 \hspace{3cm} \textbf{Name(s)}: J. Doe 

\begin{center}
PROJECT RUBRIC
\end{center}

\begin{tabular}{C | L | L | L | L | L | E}
  & & \textbf{Lacking} & \textbf{Transitional} & \textbf{Competent} & \textbf{Sophisticated} & \textbf{SCORE} \\ \hline
 \textbf{Analysis} & \textit{Ability to correctly apply a range of ML methods to the data. Should include a mixture of big- picture results (i.e., error rates) and examples (i.e., negative examples)} &
 Make major mistakes in applying or interpreting one or more models. Failure to use methods from the course for the analysis. & 
 A small number of isolated errors that affect the analysis, with some correctly applied methods and conclusions. &
 Generally correct application and understanding of the methods. May have a small number of minor issues in interpretation. &
 Correct application and interpretation of the methods in a way that demonstrates deep understanding of the techniques. &
 {\large \textbf{\magenta{30}}} \\
&&\textbf{[0]}&\textbf{[20]}&\textbf{[25]}&\textbf{[30]}& \\
 \hline
 \textbf{Synthesis} & \textit{Produce a clear synthesis of the findings and big picture of what was learned from the data.} &
 Presentation lacks a coherent synthesis or the synthesis is largely unsupported by the data. & 
 A data-oriented synthesis was given, but it is overly specific, somewhat unclear, or only partially supported by the data. &
 The synthesis of the
analyses is generally supported by the data and captures a reasonable amount of the findings in the analysis. &
 In addition to being clear and supported by the data, the synthesis shows a sophisticated understanding of the data and methods. &
  {\large \textbf{\magenta{20}}}\\
&&\textbf{[0]}&\textbf{[10]}&\textbf{[15]}&\textbf{[20]}& \\
 \hline
 \textbf{Creativity} & \textit{Show an original, thoughtful, and/or unique, way of presenting or understanding the data.} &
 The analysis/slides are repetitive, fail to show the data in different ways, and do not try to tell a story. & 
 There was a clear attempt at being creative or engaging with the presentation, but the scope or execution was lacking. &
 A succesfully executed and interesting way of presenting or understanding the data (e.g., interesting plot, custom code, creative slides). &
 There is a particularly creative element of the project that clearly extends our understanding of the data. &
  {\large \textbf{\magenta{10}}} \\
&&\textbf{[0]}&\textbf{[4]}&\textbf{[8]}&\textbf{[10]}& \\
 \hline
 \textbf{Slides} & \textit{Produce clean, professional-looking slides that balance simplicity with conveying information.} &
 The slides are difficult to read, do not follow the instructions, or include too many typos and errors. & 
 Slides generally follow the guidelines in the instructions but may contain too much/too little information, extensive typos, or not align well with the presentation. &
 The slides follow the instructions, are clear and easy to read, and contain a good balance of information and simplicity.
 &
 --- &
  {\large \textbf{\magenta{20}}} \\
&&\textbf{[0]}&\textbf{[10]}&\textbf{[20]}&& \\
 \hline
 \textbf{Communication} & \textit{Give a presentation that clearly presents the material. It should align with the slides, be the correct length, and demonstrate that the presentation was practiced.} &
 The presentation lacks coherence and is not particularly polished. May go significantly over or under the allotted time. & 
 It was clear that there was an attempt to practice the presentation but the execution is lacking (i.e., timing is off, confusion on the slides, group members repeating each other). &
 The presentation of the slides is clear and polished. The group members have dedicated time to practicing the material and are able to fluidly communicate their results to the class. &
 --- &
  {\large \textbf{\magenta{15}}} \\
&&\textbf{[0]}&\textbf{[10]}&\textbf{[15]}&& \\
 \hline
\end{tabular}

\hfill \textbf{Total:} {\large \textbf{\magenta{95}}}

\textbf{Something I Liked:} Some nice things I will write about your work.

\textbf{Something to work on:} Some suggestions for improvement on the next project.

\textbf{Grading notes:} Any notes about where I took off points above, if unclear.

\end{document}
